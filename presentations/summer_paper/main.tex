\documentclass{beamer}
\usepackage{pgfpages}
\usepackage{url}
\usepackage[round]{natbib}

\setbeameroption{show notes on second screen}
\setbeamertemplate{caption}[numbered]

\urldef{\incidentDashboard}\url{https://julianbarg.shinyapps.io/incident_dashboard/}

\title{Valuing Peace and Quiet: the Effect of Expansion Mode on Learning toward Subordinate Goals}
\author{Julian Barg}
\institute{Ivey Business School}
\date{Sept 27, 2019}
\logo{\includegraphics[scale=0.08]{resources/ivey_logo.jpg}}

\begin{document}
	
\frame{\titlepage}

\begin{frame}
	\frametitle{Outline}
	\tableofcontents
\end{frame}

\begin{frame}
	\frametitle{Summary}
	\begin{block}{Findings}
		\begin{itemize}
			\item Organization-wide strategic changes will disrupt learning that has occurred in suborganizations.
			\note{This is what expansion mode refers to in the title, not greenfield or M\&A (although there are M\&As in the dataset).\\\medskip}
			\item Thus, when "nothing happens", something does happen: organizational learning.
			\item From a learning perspective, it is better to take a long-term perspective on strategic adjustments.
				\begin{itemize}
					\item Implementing change in a less disruptive fashion.
				\end{itemize}
				\note{"Learning perspective" matters, because from a technology perspective, the conclusion might be different.}
			\item Depending on the context, implications for organizational performance (environmental and economic).
			\note{In a temporality-inspired learning research fashion. Depending on the type of organization - where mistakes threaten survival, very relevant.}
		\end{itemize}
	\end{block}
\end{frame}

\section{Cases}

\begin{frame}
	\frametitle{Cases}
	\begin{block}{High-profile bankruptcies related to pipeline incidents}
		\begin{itemize}
			\item Greka Energy/HVI Cat Canyon (1996, 2005, 2019)
			\note{Greka: History of oil spills, settlements relatively unexpensive, e.g., \$12 million fine caused bankruptcy in 2019, ironically, CEO once stated company motto as "Working for profits".\medbreak}
			\item Pacific Gas and Electric Company (2001, 2019)
			\note{PG\&E: After 2010 explosion with 8 deaths, paid \$300m in fines, \$400m in refunds, \$850m for upgrades, and \$500m in settlements.\medbreak}
			\item{EdgeMark (2019)}
			\note{EdgeMark: Energy Transfer pipeline explosion led to EdgeMark Bankruptcy. Ontario Teachers' Pension fund held minority stake.}
		\end{itemize}
	\end{block}
\end{frame}

\begin{frame}
	\frametitle{Cases}
	\incidentDashboard
	\note{"This is a tool I use for my research." Just briefly show the scale.}
\end{frame}

\section{Learning}

\begin{frame}
	\frametitle{Organizational forgetting: Definitions}
	\begin{itemize}
		\item \textit{Organizational learning}: "change in the organization's knowledge that occurs as a function of experience" \citep[p. 31]{Argote2013}.
		\item \textit{Organizational forgetting}: "the loss, voluntary or otherwise, of organizational knowledge [which] often leads to a change in organizational capabilities because of the absence of some piece of knowledge" \citep[p. 1606]{DeHolan2004}.
		\note{Both forgetting and disruptions are thought to be sometimes positive, too, but I will not cover that here.\medbreak Disruption also sometimes desdribed as merely "interacting" with forgetting (Anderson \& Lewis 2014).\medbreak}
		\item \textit{Disruptions}: "organizational change-inducing events" \citep[p. 362]{Anderson2014}
		\note{Will have examples of disruptions in the next couple of slides.}
	\end{itemize}
\end{frame}

\begin{frame}
	\frametitle{\insertsection}
	\framesubtitle{Learning curve}
	\begin{figure}
		\includegraphics[height=5cm]{resources/learning_curve1.png}
		\caption{Relation between assembly hours per aircraft and cumulative number produced. Units omitted. From: \citet[p. 921]{Argote1990}}
		\note{You have probably seen this before.}
	\end{figure}
\end{frame}

\begin{frame}
	\frametitle{\insertsection}
	\framesubtitle{Disrupted learning}
	\begin{figure}
		\includegraphics[height=5cm]{resources/learning_curve2.png}
		\caption{L-1011 Production: Direct Labor Requirement and Yearly Output. From: \citet[p. 1039]{Benkard2000}}
	\end{figure}
	\note{We see more than in the last picture. This is the production of the Lockheed L-1011. Famous for being an amazing plane, but a commercial failure. What happend here, around 150 units? Economic downturn, production was disrupted. Then, company decided the only way to break even is to produce a certain number of planes, but that did not happen.}
\end{frame}

\begin{frame}
	\frametitle{\insertsection}
	\framesubtitle{Disrupted learning}
	\begin{figure}
		\includegraphics[height=5cm]{resources/learning_curve3.jpg}
		\caption{Traditional Learning Curve: All 238 (log-log). From: \citet[p. 1046]{Benkard2000}}
	\end{figure}
	\note{This is what actually happened. After Lockheed increased production again, the number of man-hours per plane actually increased for a while, before the price decreased again.}
\end{frame}

\begin{frame}
	\begin{block}{}
		\Huge Organizational forgetting
		\bigbreak		
		\normalsize Also known as knowledge depreciation
		\note{The Lockheed is a pretty famous example for this.}
	\end{block}{}
\end{frame}

\begin{frame}
	\frametitle{Organizational forgetting}
	\begin{itemize}
		\item Turnover \citep{DeHolan2004,Rao2006}
		\note{When a large number and/or key employees leave a company, and knowledge and/or networks are lost as a result.}
		\item Loss of recorded knowledge
		\item Technology becoming irrelevant
		\item Decay of social networks \citep{Argote2013_3,Argote1990,Thompson2007}
		\item Disruption
		\item 
		\begin{itemize}
			\item Internal
				\begin{itemize}
					\item Disruption of regular production 
					\item Technology \citep{Amburgey1993,Edmondson2001}
					\item Restructuring/reorganization/layoffs (or even hirings) \citep{Benkard2000,Anderson2014}
						\note {Which is also acknowledges in the wider learning literature.\medbreak}
				\end{itemize}
			\item External
				\begin{itemize}
					\item Economic cycles \citep{Rockart2019}
					\item Natural disasters \citep{Anderson2014}
				\end{itemize}
		\end{itemize}
	\end{itemize}
	\note{Organizational disruption is special case of organizational forgetting. Disruption can stem from inside, or outside of the organization.}
\end{frame}




\begin{frame}
	\frametitle{Research Question}
	How does organizational learning and forgetting play out in the context of strategic adjustment and hierarchies?
	\note{Let me show you the phenomenon I have been working on, what I believe is going on, and how that generalizes.}
\end{frame}

\input{sections/data}

\begin{frame}[allowframebreaks]
	\frametitle{Bibliography}
	\bibliography{bibliography}
	\bibliographystyle{chicago}
\end{frame}

\end{document}