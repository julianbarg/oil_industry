\section{Conclusion}

\begin{frame}
	\frametitle{Summary}
	\begin{block}{Findings}
		\begin{itemize}
			\item Organization-wide strategic changes will disrupt learning that has occurred in suborganizations.
			\note{This is what expansion mode refers to in the title, not greenfield or M\&A (although there are M\&As in the dataset).\\\medskip}
			\item Thus, when "nothing happens", something does happen: organizational learning.
			\item From a learning perspective, it is better to take a long-term perspective on strategic adjustments.
			\begin{itemize}
				\item Implementing change in a less disruptive fashion.
			\end{itemize}
			\note{"Learning perspective" matters, because from a technology perspective, the conclusion might be different.}
			\item Depending on the context, implications for organizational performance (environmental and economic).
			\note{In a temporality-inspired learning research fashion. Depending on the type of organization - where mistakes threaten survival, very relevant.}
		\end{itemize}
	\end{block}
\end{frame}

\begin{frame}
	\frametitle{Summary}
	\begin{block}{Limitations}
		\begin{itemize}
			\item Not controling for revenue, profitability, number of employees yet.
			\note{Could merge in business unit level data from compustat.\medskip}
			\item Results driven by a number of outliers.
			\note{Already remove most extreme outliers, but have not analyzed this issue further - robustness check?\medskip}
			\item Further disentangle effect of technology vs. strategy.
			\note{Empirical analysis of the two - which dominates?}
		\end{itemize}
	\end{block}
\end{frame}