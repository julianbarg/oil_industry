\section{Hypothesis development}

\begin{frame}
	\Large So what do I think is happening in our context?
\end{frame}

\begin{frame}
	\frametitle{Hypothesis development}
	\begin{itemize}
		\item Organizations operate pipelines, part of the organization is in charge of safety.
		\note{Organization can be in the business of pipelines, or extraction. Usually a dedicated department in charge of pipeline safety.\medskip}
		\pause
		\item HQ makes strategic decisions that impact the whole organizations.
		\pause
		\item In many scenarios, the strategic decision can be a source of disruption.
		\note{For other parts of the organization.\medskip}
		\pause
		\item In the context of pipelines, we can track decision making very well, because we know the scope of the organization/suborganization company very well.
		\note{So that is what we are going to do. We track the scope of the organization, the expansion mode (sporadic or even across time), and the impact on pipeline safety.}
	\end{itemize}
\end{frame}

\begin{frame}
	\frametitle{Hypothesis development}
	\begin{block}{Hypothesis 1}
		Strategic adjustments from the parent organization disrupt the learning of suborganizations.
	\end{block}
	\pause
	
	\begin{block}{Hypothesis 2}
		Phases of stability, with less strategic adjustments being initiated by the parent organization, allow for the suborganizations to learn.
	\end{block}
	\pause
	
	\begin{block}{Hypothesis 1c \small (\textbf{C}ompeting)}
		Strategic adjustments act as external shocks to suborganizations, and induce additional learning.
	\end{block}
\end{frame}