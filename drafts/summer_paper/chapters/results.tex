\section{Results}

{\centering
	//////////////////////////////////////////////////////////////////////////////
	
	[Insert Table 1 approximately here]
	
	//////////////////////////////////////////////////////////////////////////////\par
}

\begin{table}
	\rotatebox{90}{
		\begin{minipage}{1.4 \textwidth}
			\begin{table}[H]
\centering
\resizebox{\linewidth}{!}{
\begin{tabular}{llllllllllllllllllllll}
\toprule
  & Mean & SD & Incidts & Adj. & Adj sq. & Miles add & Consolid. & MA & MA t-1 & Miles Crude & Age Crude & MxA Crude & M. HVL & A. HVL & MxA HVL & M. Non-HVL & A. Non-HVL & MxA Non-HVL & No Crude & No HVL & No Non-HVL\\
\midrule
Incidts & 4.71 & 7.68 & 1 &  &  &  &  &  &  &  &  &  &  &  &  &  &  &  &  &  & \\
Adj. & 0.16 & 0.59 & -0.06 & 1 &  &  &  &  &  &  &  &  &  &  &  &  &  &  &  &  & \\
Adj sq. & 0.38 & 2.62 & -0.06 & 0.93 & 1 &  &  &  &  &  &  &  &  &  &  &  &  &  &  &  & \\
Miles add & 0.25 & 0.41 & -0.04 & 0.31 & 0.16 & 1 &  &  &  &  &  &  &  &  &  &  &  &  &  &  & \\
Consolid. & 0.72 & 2.71 & -0.1 & 0.06 & -0.02 & 0.39 & 1 &  &  &  &  &  &  &  &  &  &  &  &  &  & \\
\addlinespace
MA & 0.02 & 0.14 & 0.19 & -0.02 & -0.02 & 0.01 & -0.02 & 1 &  &  &  &  &  &  &  &  &  &  &  &  & \\
MA t-1 & 0.02 & 0.14 & 0.19 & -0.03 & -0.02 & 0.02 & -0.02 & 0.24 & 1 &  &  &  &  &  &  &  &  &  &  &  & \\
Miles Crude & 541.85 & 1254.53 & 0.48 & -0.06 & -0.05 & -0.01 & -0.05 & 0.27 & 0.21 & 1 &  &  &  &  &  &  &  &  &  &  & \\
Age Crude & 21.03 & 19.99 & 0.07 & -0.11 & -0.09 & -0.04 & -0.15 & 0.03 & 0 & 0.32 & 1 &  &  &  &  &  &  &  &  &  & \\
MxA Crude & 20003.45 & 49261.14 & 0.45 & -0.06 & -0.05 & -0.01 & -0.05 & 0.28 & 0.11 & 0.98 & 0.36 & 1 &  &  &  &  &  &  &  &  & \\
\addlinespace
M. HVL & 954.78 & 2299.51 & 0.6 & -0.04 & -0.03 & -0.05 & -0.08 & 0 & 0.04 & -0.01 & -0.14 & -0.01 & 1 &  &  &  &  &  &  &  & \\
A. HVL & 19.81 & 17.91 & 0.27 & 0.09 & 0.08 & 0.05 & -0.06 & 0.09 & 0.11 & 0.17 & -0.13 & 0.16 & 0.29 & 1 &  &  &  &  &  &  & \\
MxA HVL & 30994.3 & 67449.59 & 0.61 & -0.03 & -0.02 & -0.05 & -0.08 & -0.01 & 0.05 & -0.01 & -0.11 & 0.01 & 0.99 & 0.34 & 1 &  &  &  &  &  & \\
M. Non-HVL & 1631.07 & 3280.34 & 0.57 & -0.08 & -0.05 & -0.08 & -0.1 & 0.05 & 0.03 & 0.15 & 0.13 & 0.16 & 0.25 & 0.2 & 0.31 & 1 &  &  &  &  & \\
A. Non-HVL & 23.37 & 18.76 & 0.33 & 0.02 & 0.03 & 0.01 & -0.11 & 0.03 & 0.01 & 0.12 & 0.06 & 0.12 & 0.23 & 0.2 & 0.24 & 0.39 & 1 &  &  &  & \\
\addlinespace
MxA Non-HVL & 67228.73 & 137336.75 & 0.58 & -0.07 & -0.04 & -0.07 & -0.09 & 0.05 & -0.01 & 0.11 & 0.12 & 0.14 & 0.26 & 0.2 & 0.28 & 0.98 & 0.41 & 1 &  &  & \\
No Crude & 0.39 & 0.49 & -0.07 & 0.15 & 0.14 & 0.01 & 0.16 & -0.06 & -0.07 & -0.35 & -0.8 & -0.33 & 0.15 & 0.13 & 0.13 & -0.05 & 0 & -0.01 & 1 &  & \\
No HVL & 0.38 & 0.49 & -0.29 & -0.12 & -0.1 & -0.14 & -0.07 & -0.11 & -0.12 & -0.16 & 0.26 & -0.16 & -0.33 & -0.84 & -0.36 & -0.18 & -0.18 & -0.2 & -0.29 & 1 & \\
No Non-HVL & 0.3 & 0.46 & -0.27 & 0.09 & 0.09 & 0.11 & 0.18 & -0.06 & -0.02 & -0.13 & 0.03 & -0.13 & -0.18 & -0.18 & -0.19 & -0.33 & -0.78 & -0.31 & 0.03 & 0.13 & 1\\
\bottomrule
\end{tabular}}
\end{table}

		\end{minipage}
	}
	\caption{}
\end{table}

{\centering
	//////////////////////////////////////////////////////////////////////////////
	
	[Insert Table 2 approximately here]
	
	//////////////////////////////////////////////////////////////////////////////\par
}

\begin{table}
	\rotatebox{90}{
		\begin{minipage}{1.4 \textwidth}
			\begin{table}[H]
\centering
\resizebox{\linewidth}{!}{
\begin{tabular}{llllllllllllllllllllllllllllllllll}
\toprule
  & Mean & SD & Incidts & Adj. & Adj sq. & Miles add & Consolid. & MA & MA t-1 & Crude 40s & Crude 50s & Crude 60s & Crude 70s & Crude 80s & Crude 90s & Crude 00s & Crude 10s & HVL 40s & HVL 50s & HVL 60s & HVL 70s & HVL 80s & HVL 90s & HLV 00s & HVL 10s & Non-HVL 40s & Non-HVL 50s & Non-HVL 60s & Non-HVL 70s & Non-HVL 80s & Non-HVL 90s & Non-HVL 00s & Non-HVL 10s\\
\midrule
Incidts & 4.71 & 7.68 & 1 &  &  &  &  &  &  &  &  &  &  &  &  &  &  &  &  &  &  &  &  &  &  &  &  &  &  &  &  &  & \\
Adj. & 0.16 & 0.59 & -0.06 & 1 &  &  &  &  &  &  &  &  &  &  &  &  &  &  &  &  &  &  &  &  &  &  &  &  &  &  &  &  & \\
Adj sq. & 0.38 & 2.62 & -0.06 & 0.93 & 1 &  &  &  &  &  &  &  &  &  &  &  &  &  &  &  &  &  &  &  &  &  &  &  &  &  &  &  & \\
Miles add & 0.25 & 0.41 & -0.04 & 0.31 & 0.16 & 1 &  &  &  &  &  &  &  &  &  &  &  &  &  &  &  &  &  &  &  &  &  &  &  &  &  &  & \\
Consolid. & 0.72 & 2.71 & -0.1 & 0.06 & -0.02 & 0.39 & 1 &  &  &  &  &  &  &  &  &  &  &  &  &  &  &  &  &  &  &  &  &  &  &  &  &  & \\
\addlinespace
MA & 0.02 & 0.14 & 0.19 & -0.02 & -0.02 & 0.01 & -0.02 & 1 &  &  &  &  &  &  &  &  &  &  &  &  &  &  &  &  &  &  &  &  &  &  &  &  & \\
MA t-1 & 0.02 & 0.14 & 0.19 & -0.03 & -0.02 & 0.02 & -0.02 & 0.24 & 1 &  &  &  &  &  &  &  &  &  &  &  &  &  &  &  &  &  &  &  &  &  &  &  & \\
Crude 40s & 90.48 & 283.05 & 0.23 & -0.06 & -0.04 & -0.06 & -0.06 & 0.11 & 0.08 & 1 &  &  &  &  &  &  &  &  &  &  &  &  &  &  &  &  &  &  &  &  &  &  & \\
Crude 50s & 196.52 & 465.59 & 0.33 & -0.09 & -0.06 & -0.07 & -0.07 & 0.14 & 0.08 & 0.71 & 1 &  &  &  &  &  &  &  &  &  &  &  &  &  &  &  &  &  &  &  &  &  & \\
Crude 60s & 111.21 & 332.14 & 0.35 & -0.07 & -0.04 & -0.06 & -0.06 & 0.17 & 0.13 & 0.78 & 0.69 & 1 &  &  &  &  &  &  &  &  &  &  &  &  &  &  &  &  &  &  &  &  & \\
\addlinespace
Crude 70s & 88.78 & 188.15 & 0.14 & -0.09 & -0.06 & -0.09 & -0.08 & 0.01 & -0.01 & 0.06 & 0.33 & 0.25 & 1 &  &  &  &  &  &  &  &  &  &  &  &  &  &  &  &  &  &  &  & \\
Crude 80s & 80.42 & 358.7 & 0.05 & -0.05 & -0.03 & -0.06 & -0.05 & -0.01 & -0.01 & 0.07 & 0.19 & 0.05 & 0.52 & 1 &  &  &  &  &  &  &  &  &  &  &  &  &  &  &  &  &  &  & \\
Crude 90s & 141.19 & 450.95 & 0.25 & -0.06 & -0.04 & -0.08 & -0.05 & 0.3 & 0.18 & 0.54 & 0.47 & 0.53 & 0.13 & 0.1 & 1 &  &  &  &  &  &  &  &  &  &  &  &  &  &  &  &  &  & \\
Crude 00s & 67.36 & 200.6 & 0.1 & -0.06 & -0.04 & -0.05 & -0.04 & 0.04 & 0.04 & 0.2 & 0.28 & 0.23 & 0.27 & 0.41 & 0.29 & 1 &  &  &  &  &  &  &  &  &  &  &  &  &  &  &  &  & \\
Crude 10s & 32.24 & 122.97 & 0.16 & -0.04 & -0.03 & -0.05 & -0.03 & -0.02 & 0.09 & 0.01 & 0.11 & 0.08 & 0.26 & 0.07 & 0.06 & 0.27 & 1 &  &  &  &  &  &  &  &  &  &  &  &  &  &  &  & \\
\addlinespace
HVL 40s & 49.73 & 164.69 & 0.39 & -0.05 & -0.03 & -0.03 & -0.06 & -0.02 & 0.07 & 0.49 & 0.5 & 0.49 & 0.04 & 0 & 0.01 & -0.04 & 0.05 & 1 &  &  &  &  &  &  &  &  &  &  &  &  &  &  & \\
HVL 50s & 148.2 & 422.6 & 0.32 & 0.03 & 0 & 0.02 & -0.07 & 0.1 & 0.12 & 0.32 & 0.33 & 0.32 & -0.01 & -0.01 & 0.1 & -0.02 & -0.02 & 0.47 & 1 &  &  &  &  &  &  &  &  &  &  &  &  &  & \\
HVL 60s & 512.26 & 1455.51 & 0.57 & 0 & -0.02 & -0.01 & -0.06 & -0.02 & 0 & 0.03 & 0.06 & 0.05 & -0.02 & -0.01 & -0.02 & -0.04 & 0 & 0.56 & 0.37 & 1 &  &  &  &  &  &  &  &  &  &  &  &  & \\
HVL 70s & 511.79 & 1069.86 & 0.47 & 0.02 & 0.01 & -0.03 & -0.1 & -0.03 & -0.04 & 0.05 & 0.13 & 0.05 & -0.06 & -0.05 & 0.03 & -0.1 & 0.07 & 0.48 & 0.43 & 0.68 & 1 &  &  &  &  &  &  &  &  &  &  &  & \\
HVL 80s & 244.6 & 909.02 & 0.48 & 0 & -0.02 & 0 & -0.04 & -0.02 & -0.03 & 0.01 & -0.01 & -0.02 & -0.04 & 0.06 & -0.06 & -0.03 & -0.04 & 0.45 & 0.22 & 0.87 & 0.7 & 1 &  &  &  &  &  &  &  &  &  &  & \\
\addlinespace
HVL 90s & 294.94 & 1008.73 & 0.51 & -0.01 & -0.02 & -0.01 & -0.05 & 0 & -0.02 & 0.1 & 0.08 & 0.09 & -0.08 & -0.02 & 0 & -0.06 & -0.06 & 0.53 & 0.28 & 0.87 & 0.71 & 0.96 & 1 &  &  &  &  &  &  &  &  &  & \\
HLV 00s & 136.4 & 531.01 & 0.35 & -0.03 & -0.03 & -0.01 & -0.02 & -0.01 & 0 & -0.07 & -0.07 & -0.07 & -0.1 & -0.05 & -0.07 & -0.08 & -0.05 & 0.26 & 0.19 & 0.64 & 0.51 & 0.69 & 0.66 & 1 &  &  &  &  &  &  &  &  & \\
HVL 10s & 67.38 & 407.69 & 0.29 & -0.04 & -0.02 & -0.04 & -0.02 & 0.05 & 0.19 & -0.02 & -0.02 & -0.01 & -0.05 & -0.03 & -0.04 & -0.04 & 0.05 & 0.2 & 0.27 & 0.26 & 0.3 & 0.36 & 0.37 & 0.34 & 1 &  &  &  &  &  &  &  & \\
Non-HVL 40s & 154.34 & 597.85 & 0.36 & -0.03 & -0.02 & -0.03 & -0.05 & 0 & -0.02 & -0.03 & 0.03 & 0.02 & 0.07 & -0.05 & -0.06 & -0.07 & 0.05 & 0.04 & 0.12 & 0.1 & 0.11 & -0.01 & 0.02 & -0.01 & -0.02 & 1 &  &  &  &  &  &  & \\
Non-HVL 50s & 430.49 & 1282.41 & 0.44 & -0.05 & -0.03 & -0.03 & -0.07 & 0.01 & 0 & 0.17 & 0.25 & 0.24 & 0.14 & -0.01 & 0.03 & -0.04 & 0.12 & 0.24 & 0.31 & 0.15 & 0.26 & 0 & 0.05 & -0.01 & -0.02 & 0.87 & 1 &  &  &  &  &  & \\
\addlinespace
Non-HVL 60s & 435.28 & 1049.93 & 0.51 & -0.09 & -0.05 & -0.08 & -0.09 & 0.02 & 0.01 & 0.14 & 0.29 & 0.17 & 0.1 & -0.04 & 0.11 & -0.03 & 0.13 & 0.21 & 0.2 & 0.09 & 0.38 & 0.01 & 0.06 & 0.02 & -0.03 & 0.5 & 0.61 & 1 &  &  &  &  & \\
Non-HVL 70s & 305.52 & 665.39 & 0.52 & -0.1 & -0.06 & -0.09 & -0.1 & 0.08 & 0.05 & 0.06 & 0.17 & 0.11 & 0.08 & -0.07 & 0.03 & -0.09 & 0.12 & 0.26 & 0.1 & 0.27 & 0.34 & 0.2 & 0.25 & 0.07 & 0.05 & 0.51 & 0.6 & 0.65 & 1 &  &  &  & \\
Non-HVL 80s & 174.22 & 503.67 & 0.33 & -0.07 & -0.04 & -0.08 & -0.08 & -0.01 & 0 & 0.26 & 0.5 & 0.25 & 0.42 & 0.61 & 0.1 & 0.22 & 0.16 & 0.38 & 0.21 & 0.12 & 0.36 & 0.13 & 0.12 & 0.01 & -0.03 & 0.04 & 0.23 & 0.55 & 0.35 & 1 &  &  & \\
Non-HVL 90s & 200.85 & 562.18 & 0.21 & -0.07 & -0.04 & -0.1 & -0.07 & 0.12 & 0.11 & 0.15 & 0.31 & 0.19 & 0.25 & 0.24 & 0.35 & 0.11 & 0.2 & 0.2 & 0.14 & 0.08 & 0.33 & 0.01 & 0.04 & -0.04 & -0.04 & 0.19 & 0.43 & 0.43 & 0.42 & 0.53 & 1 &  & \\
Non-HVL 00s & 119.97 & 313.53 & 0.43 & -0.09 & -0.05 & -0.08 & -0.09 & 0.19 & 0.11 & 0.46 & 0.67 & 0.45 & 0.19 & -0.02 & 0.45 & 0.13 & 0.16 & 0.34 & 0.25 & 0.07 & 0.34 & 0 & 0.07 & -0.02 & -0.04 & 0.27 & 0.47 & 0.63 & 0.49 & 0.52 & 0.49 & 1 & \\
\addlinespace
Non-HVL 10s & 14.97 & 100.05 & 0.06 & -0.04 & -0.02 & -0.02 & -0.02 & -0.02 & 0.01 & 0.06 & 0.09 & 0.08 & 0.01 & 0.02 & 0.02 & 0.15 & 0.05 & 0.08 & 0.06 & 0 & 0.01 & -0.01 & 0 & -0.01 & 0.05 & 0 & 0.08 & 0.06 & 0 & 0.09 & 0.18 & 0.12 & 1\\
\bottomrule
\end{tabular}}
\end{table}

		\end{minipage}
	}
	\caption{}
\end{table}

We ran a fixed effects model with clustered standard errors to estimate the effects. Before selecting that model specification, we also ran a Hausman test to ensure that a fixed effects model is an appropriate choice. The linear effect of adjustments is significant at the 0.1 confidence level for models 2, 4,5, and 6, whereas the squared effect is significant at the 0.1 confidence interval only for model 2. For the other models, the standard error does place the results close to the 0.1 confidence level also (see Table 3 and Table 4). To get an impression of the effect of the adjustment variable at above and below 0, we plotted out both the linear and squared effect for model 2, which would show the most easily interpretable trend due to the significance level of both the the linear and squared effect (see Figure 4) -- the results may not be generalizable, as the squared effect is not significant in the other models. The clear negative trend for negative values implies that when a company enters a phase of stability, the rate of significant incidents does decrease, supporting \textit{Hypothesis 2}. For the positive range, which describes an uptick in strategic adjustments, the 95\% confidence interval includes both positive and negative values -- there is no clear evidence that these adjustments from the parent organizations disrupt or setback the learning toward pipeline safety. \textit{Hypothesis 1} does not have clear support. On the other hand, the results also do not allow us to conclude that strategic adjustments act as shocks that facilitate additional learning (competing \textit{Hypothesis 1c}).

{\centering
	//////////////////////////////////////////////////////////////////////////////
	
	[Insert Table 3 approximately here]
	
	//////////////////////////////////////////////////////////////////////////////\par
}

\begin{table}
	{\renewcommand\normalsize{\tiny}%
		\normalsize
	\begin{tabular}{llll}
\toprule
{} &                        Model 1 &                         Model 2 &                         Model 3 \\
\midrule
Adjustments         &       $\makecell{1.5\\(1.21)}$ &   $\makecell{1.93^{*}\\(1.15)}$ &       $\makecell{1.41\\(1.19)}$ \\
Adjustments sq.     &     $\makecell{-0.46\\(0.33)}$ &  $\makecell{-0.54^{*}\\(0.31)}$ &      $\makecell{-0.42\\(0.32)}$ \\
Miles add           &      $\makecell{0.17\\(0.59)}$ &                                 &        $\makecell{0.1\\(0.59)}$ \\
Consolidation       &       $\makecell{0.07\\(0.1)}$ &                                 &        $\makecell{0.08\\(0.1)}$ \\
M and A             &      $\makecell{1.53\\(1.92)}$ &  $\makecell{3.75^{**}\\(1.64)}$ &       $\makecell{1.75\\(2.02)}$ \\
M and A t-1         &     $\makecell{-1.58\\(1.14)}$ &      $\makecell{-1.21\\(0.83)}$ &      $\makecell{-1.36\\(1.16)}$ \\
Miles Crude         &        $\makecell{0.0\\(0.0)}$ &         $\makecell{0.0\\(0.0)}$ &         $\makecell{0.0\\(0.0)}$ \\
Age Crude           &  $\makecell{0.04^{*}\\(0.02)}$ &       $\makecell{0.01\\(0.02)}$ &   $\makecell{0.04^{*}\\(0.02)}$ \\
Miles x Age Crude   &       $\makecell{-0.0\\(0.0)}$ &   $\makecell{-0.0^{**}\\(0.0)}$ &        $\makecell{-0.0\\(0.0)}$ \\
Miles HVL           &      $\makecell{0.01\\(0.01)}$ &        $\makecell{0.0\\(0.01)}$ &       $\makecell{0.01\\(0.01)}$ \\
Age HVL             &      $\makecell{0.04\\(0.03)}$ &   $\makecell{0.04^{*}\\(0.03)}$ &       $\makecell{0.04\\(0.03)}$ \\
Miles x Age HVL     &       $\makecell{-0.0\\(0.0)}$ &        $\makecell{-0.0\\(0.0)}$ &        $\makecell{-0.0\\(0.0)}$ \\
Miles Non-HVL       &        $\makecell{0.0\\(0.0)}$ &         $\makecell{0.0\\(0.0)}$ &         $\makecell{0.0\\(0.0)}$ \\
Age Non-HVL         &      $\makecell{-0.0\\(0.03)}$ &        $\makecell{0.0\\(0.02)}$ &      $\makecell{-0.01\\(0.03)}$ \\
Miles x Age Non-HVL &       $\makecell{-0.0\\(0.0)}$ &        $\makecell{-0.0\\(0.0)}$ &        $\makecell{-0.0\\(0.0)}$ \\
No Crude            &  $\makecell{-8.66^{*}\\(4.5)}$ &      $\makecell{-1.78\\(2.71)}$ &  $\makecell{-8.63^{*}\\(4.39)}$ \\
No HVL              &      $\makecell{1.68\\(1.22)}$ &       $\makecell{0.98\\(0.85)}$ &       $\makecell{1.69\\(1.32)}$ \\
No Non-HVL          &     $\makecell{-3.19\\(2.94)}$ &       $\makecell{-1.59\\(2.4)}$ &      $\makecell{-2.87\\(3.02)}$ \\
2007                &                                &                                 &      $\makecell{-0.19\\(0.38)}$ \\
2008                &                                &                                 &       $\makecell{0.03\\(0.57)}$ \\
2009                &                                &                                 &        $\makecell{0.3\\(0.65)}$ \\
2010                &                                &                                 &        $\makecell{0.43\\(0.7)}$ \\
2011                &                                &                                 &       $\makecell{0.44\\(0.62)}$ \\
2012                &                                &                                 &       $\makecell{0.56\\(0.69)}$ \\
Constant            &      $\makecell{2.12\\(2.42)}$ &      $\makecell{-1.83\\(2.42)}$ &       $\makecell{1.89\\(2.39)}$ \\
Groups              &                             69 &                              70 &                              69 \\
Observations        &                            401 &                             474 &                             401 \\
R-sq within         &                           0.33 &                            0.32 &                            0.33 \\
R-sq between        &                            0.3 &                            0.43 &                             0.3 \\
R-sq overall        &                           0.34 &                            0.42 &                            0.33 \\
\bottomrule
*: Significant at the 0.9 level. & **: Significant at the 0.95 level. & ***: Significant at the 0.99 level.
\end{tabular}
}
	\caption{}
\end{table}

{\centering
	//////////////////////////////////////////////////////////////////////////////
	
	[Insert Table 4 approximately here]
	
	//////////////////////////////////////////////////////////////////////////////\par
}

\begin{table}
	{\renewcommand\normalsize{\tiny}%
		\normalsize
		\begin{tabular}{ll}

\begin{tabular}{llll}
\toprule
{} &                           Model 4 &                          Model 5 &                           Model 6 \\
\midrule
Adjustments         &     $\makecell{1.22^{*}\\(0.65)}$ &    $\makecell{1.29^{*}\\(0.77)}$ &     $\makecell{1.24^{*}\\(0.66)}$ \\
Adjustments sq.     &        $\makecell{-0.22\\(0.14)}$ &       $\makecell{-0.26\\(0.17)}$ &        $\makecell{-0.21\\(0.14)}$ \\
Miles add           &    $\makecell{-0.57^{*}\\(0.33)}$ &                                  &        $\makecell{-0.48\\(0.32)}$ \\
Consolidation       &    $\makecell{0.1^{***}\\(0.03)}$ &                                  &   $\makecell{0.09^{***}\\(0.03)}$ \\
M and A             &         $\makecell{2.22\\(1.45)}$ &  $\makecell{2.31^{***}\\(0.79)}$ &          $\makecell{2.47\\(1.5)}$ \\
M and A t-1         &  $\makecell{-0.89^{***}\\(0.28)}$ &        $\makecell{1.15\\(0.77)}$ &   $\makecell{-0.96^{**}\\(0.44)}$ \\
Miles Crude 1940s   &          $\makecell{-0.0\\(0.0)}$ &         $\makecell{-0.0\\(0.0)}$ &           $\makecell{0.0\\(0.0)}$ \\
Miles Crude 1950s   &    $\makecell{0.01^{***}\\(0.0)}$ &      $\makecell{0.0^{*}\\(0.0)}$ &    $\makecell{0.01^{***}\\(0.0)}$ \\
Miles Crude 1960s   &         $\makecell{0.01\\(0.01)}$ &          $\makecell{0.0\\(0.0)}$ &         $\makecell{0.01\\(0.01)}$ \\
Miles Crude 1970s   &           $\makecell{0.0\\(0.0)}$ &          $\makecell{0.0\\(0.0)}$ &           $\makecell{0.0\\(0.0)}$ \\
Miles Crude 1980s   &         $\makecell{0.01\\(0.01)}$ &        $\makecell{0.01\\(0.01)}$ &         $\makecell{0.01\\(0.01)}$ \\
Miles Crude 1990s   &           $\makecell{0.0\\(0.0)}$ &          $\makecell{0.0\\(0.0)}$ &           $\makecell{0.0\\(0.0)}$ \\
Miles Crude 2000s   &          $\makecell{-0.0\\(0.0)}$ &      $\makecell{0.0^{*}\\(0.0)}$ &          $\makecell{-0.0\\(0.0)}$ \\
Miles Crude 2010s   &         $\makecell{-0.0\\(0.01)}$ &          $\makecell{0.0\\(0.0)}$ &          $\makecell{0.0\\(0.01)}$ \\
Miles HVL 1940s     &        $\makecell{-0.01\\(0.01)}$ &         $\makecell{0.0\\(0.01)}$ &        $\makecell{-0.01\\(0.01)}$ \\
Miles HVL 1950s     &          $\makecell{-0.0\\(0.0)}$ &          $\makecell{0.0\\(0.0)}$ &          $\makecell{-0.0\\(0.0)}$ \\
Miles HVL 1960s     &      $\makecell{0.0^{**}\\(0.0)}$ &     $\makecell{0.0^{**}\\(0.0)}$ &      $\makecell{0.0^{**}\\(0.0)}$ \\
Miles HVL 1970s     &    $\makecell{-0.0^{***}\\(0.0)}$ &   $\makecell{-0.0^{***}\\(0.0)}$ &    $\makecell{-0.0^{***}\\(0.0)}$ \\
Miles HVL 1980s     &          $\makecell{-0.0\\(0.0)}$ &         $\makecell{-0.0\\(0.0)}$ &           $\makecell{0.0\\(0.0)}$ \\
Miles HVL 1990s     &           $\makecell{0.0\\(0.0)}$ &     $\makecell{0.0^{**}\\(0.0)}$ &           $\makecell{0.0\\(0.0)}$ \\
Miles HVL 2000s     &      $\makecell{0.0^{**}\\(0.0)}$ &          $\makecell{0.0\\(0.0)}$ &      $\makecell{0.0^{**}\\(0.0)}$ \\
Miles HVL 2010s     &    $\makecell{0.02^{**}\\(0.01)}$ &          $\makecell{0.0\\(0.0)}$ &    $\makecell{0.02^{**}\\(0.01)}$ \\
\bottomrule
cont.
\end{tabular}

\begin{tabular}{llll}
\toprule
{} &                           Model 4 &                          Model 5 &                           Model 6 \\
\midrule
Miles Non-HVL 1940s &   $\makecell{-0.01^{***}\\(0.0)}$ &   $\makecell{-0.0^{***}\\(0.0)}$ &   $\makecell{-0.01^{***}\\(0.0)}$ \\
Miles Non-HVL 1950s &          $\makecell{-0.0\\(0.0)}$ &          $\makecell{0.0\\(0.0)}$ &          $\makecell{-0.0\\(0.0)}$ \\
Miles Non-HVL 1960s &           $\makecell{0.0\\(0.0)}$ &          $\makecell{0.0\\(0.0)}$ &           $\makecell{0.0\\(0.0)}$ \\
Miles Non-HVL 1970s &           $\makecell{0.0\\(0.0)}$ &          $\makecell{0.0\\(0.0)}$ &           $\makecell{0.0\\(0.0)}$ \\
Miles Non-HVL 1980s &    $\makecell{0.02^{***}\\(0.0)}$ &   $\makecell{0.02^{***}\\(0.0)}$ &    $\makecell{0.02^{***}\\(0.0)}$ \\
Miles Non-HVL 1990s &          $\makecell{0.0\\(0.01)}$ &        $\makecell{-0.0\\(0.01)}$ &          $\makecell{0.0\\(0.01)}$ \\
Miles Non-HVL 2000s &          $\makecell{-0.0\\(0.0)}$ &         $\makecell{-0.0\\(0.0)}$ &          $\makecell{-0.0\\(0.0)}$ \\
Miles Non-HVL 2010s &           $\makecell{0.0\\(0.0)}$ &     $\makecell{0.01^{*}\\(0.0)}$ &           $\makecell{0.0\\(0.0)}$ \\
2007                &                                   &                                  &         $\makecell{-0.3\\(0.32)}$ \\
2008                &                                   &                                  &         $\makecell{0.02\\(0.35)}$ \\
2009                &                                   &                                  &         $\makecell{0.24\\(0.47)}$ \\
2010                &                                   &                                  &         $\makecell{0.13\\(0.52)}$ \\
2011                &                                   &                                  &        $\makecell{-0.29\\(0.48)}$ \\
2012                &                                   &                                  &        $\makecell{-0.53\\(0.59)}$ \\
Constant            &  $\makecell{-4.21^{***}\\(1.11)}$ &  $\makecell{-2.88^{**}\\(1.23)}$ &  $\makecell{-4.21^{***}\\(1.08)}$ \\
Groups              &                                69 &                               78 &                                69 \\
Observations        &                               401 &                              624 &                               401 \\
R-sq within         &                              0.75 &                             0.62 &                              0.76 \\
R-sq between        &                               0.2 &                              0.3 &                              0.19 \\
R-sq overall        &                              0.21 &                              0.3 &                               0.2 \\
\bottomrule
*: Significant at the 0.9 level. & **: Significant at the 0.95 level. & ***: Significant at the 0.99 level.
\end{tabular}

\end{tabular}}
	\caption{}
\end{table}

{\centering
	//////////////////////////////////////////////////////////////////////////////
	
	[Insert Figure 4 approximately here]
	
	//////////////////////////////////////////////////////////////////////////////\par
}

\begin{figure}
	\includegraphics{illustrations/effect_size.png}
	\caption{}
\end{figure}
