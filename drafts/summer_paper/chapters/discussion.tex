\section{Discussion and Conclusion}
Our results imply that phases of stability facilitate learning toward subordinate goals by suborganizations, whereas strategic adjustments might disrupt learning. \textit{Hypothesis 2} regarding the positive effect of stability is supported in four of the six models we ran, while there is only limited support for \textit{Hypothesis 1} regarding disruptions. The statistical model tests for within effects, and is more robust than a traditional GLS regression would be; there is also support in one of the two models that control for year effects. While the results are not yet fully persuasive, they do indicate that at least the research direction has merit. 

There is a large literature on how organizations can learn from one-off events, such as the Deepwater Horizon disaster \citep[e.g.,][]{March1991, Haunschild2015}. The context of pipeline incidents might provide more evidence of how organizations intentionally or unintentionally (at a macro level) steer the safety record of their organizations. One could assume that similar factors play a role in both the smaller accidents which we study in this paper (and which often go unreported in the media), and the larger accidents which affect large ecosystems (and occasionally threaten the survival of organizations): if this relationship between smaller and larger accidents exists, then the empirical context of this paper would be deserving of more attention indeed.

Two related levels of analysis exist that we have not touched on so far. One is the question of the positive impact of technology, and one is that of the potential impact of accidents on organizational environmental (and social) performance and organizational survival. We have controller for technology in our empirical analysis by including variable on the age of the pipeline network. In a context where the level of change in technology is considerable, the positive impact from technological modernization might by far outweigh any negative impacts from organizational processes.  When the impacts of an observation (accident) at the extreme end are considerable on the other hand (such as the impact of the Deepwater Horizon disaster on the Gulf of Mexico), the implications are the inverse of those of technology: these rare events then need to be taken into consideration when questions of environmental impacts and survival are discussed. Environmental performance is typically measured as the environmental footprint, or in a lifecycle analysis. To assess this impact, a consulting firm will look at regular inputs and outputs. Inputs can be decreased through simple efficiency-improving measures, such as the procurement of more energy-efficient equipment, or raw materials with a lower environmental impact. More and more we live in a world however that is shaped and threatened by rare events \citep{Beck1992}. If managerial actions have some impact on the occurrence of these rare events, then we should move these actions into sharper focus. Besides, managerial action that impacts likelihood of incidents could have also prevented the bankruptcy of the pipeline operator HVI Cat Canyon that was mentioned in the introduction.

The relationship of managerial steering and long-run impacts is already studied in the literature on temporality. That literature speaks to our context: organizational outcomes can be improved when firms plan ahead and create long-term roadmaps in addition to the short and mid-term business plans that companies necessarily have \citep{Slawinski2015}. On the other hand, a very short-sighted planning approach in conjunction with frequent directional changes will lead to inferior outcomes \citep{Bansal2014, Morales-Raya2015}.

Our paper still has some shortcomings that need to be addressed. Most importantly, our dataset so far mostly contains data from the FERC. Financial data on the organizations, such as expenditure and revenue of the organizations could be added; this would allow us to isolate the effects of interest better. In particular, companies that come under financial pressure might sacrifice pipeline safety in a (myopic) bid to improve the profit margin. Compustat and other databases provide data not only at the organization, but also at the business unit level (e.g., revenue of pipeline business as reported for that sector in the annual report); merging in that data would be particularly effective. There is also more data in the dataset itself that should be sighted: which are the organizations that make many adjustments, and why? Further, in the FERC data the incidents are described in enormous detail; \citet{Park2019} for example leverages this information better than we do at the moment. Finally, our analysis is backed by limited institutional knowledge. If we were to substitute the empirical analysis with e.g., interviews, we could gain more insights on other factors that affect incident rates, and the relationships at play. On a more practical note, considering that our sample size ranges from 401 to 624, and missingness (as well as our construction of variables as multi-year averages) reduces the effective observation period to 2004-2015 (with the effective years being listed in the fixed-effect regression as a result ranging from 2007-2012), we can also do more to leverage the statistical power of the available data. Finally, some operators have only very small pipeline networks; the range of their change variables (which is measured in percent) is therefore very large. The dataset could be tailored to limit the impact of this source of noise.