{Organizational Learning, Forgetting, and Disruptions}
Strategic approaches to development vary substantially, even between seemingly similar companies. \citet{Morales-Raya2015} describes the different approaches that the Coca-Cola Company and PepsiCo have taken over the years: Coca Cola has taken more initiatives to capture new markets and generally "moved fast" where it saw a business opportunity, but often with sloppy execution. Pepsi has relied less on radical, ambitious new projects, yet, over time managed to achieve a higher market capitalization. The strategic approach that the top management of a company takes determines how well an organization can learn from new experiences, and build on existing knowledge. In the quickly moving environment of Coca-Cola, high-profile mistakes occurred at a higher rate. The three determining factors the impact of top level action on organization-wide learning are \textit{knowledge depreciation}, \textit{disruptions} and opportunities for learning, e.g., \textit{vicarious learning}.

\subsection{Knowledge Depreciation/Organizational Forgetting}

The learning curve (in its many different iterations) famously showcases how an organization over time drives down the unit cost of goods \citep{Wright1936} or services, and that the organization can over time improve other performance metrics also \citep[p. 1]{Argote2013_1}. For some time, an implicit assumption underlying this research stream has been that organizational knowledge is cumulative, that it "persists indefinitely through time" \citep[p. 57]{Argote2013_3}. If that was the case, the mode of extension should not be a concern for pipeline operators (or companies in other sectors): the attained knowledge stock would continue to ensure good performance in existing facilities or infastructure, only for the newly added assets some new learning might be required. The literature on knowledge depreciation or organizational forgetting challenges some of the conventional wisdom on learning. \citet{Argote2013_3} first probed the existence of knowledge depreciation by fitting a parameter lambda to an organization's stock of experience (measured as past output): for each time unit passed, the stock of experience from past work output would be adjusted by the factor lambda. A value of one would indicate that a unit of output produced 10 years ago would contribute as much to current efficiency as one unit of output from yesterday. A maximum likelihood estimate for lambda (when fitted to the data) of less than one indicates knowledge depreciation; in this case, it would be fair to assume that knowledge depreciated over time \citep[pp. 62ff.]{Argote2013_3}.

The literature described numerous different mechanisms of knowledge depreciation. Knowledge that resides within employees can be lost through employee turnover \citep{DeHolan2004, Rao2006}, or when an activity is not carried out regularly \citep{Ramdas2018}. \footnote{Software developers for instance consider the \textit{bus factor} or \textit{truck factor}: how robust is understanding of an ongoing project throughout the team? If one or multiple individuals were hit by the bus tomorrow, could development of the project continue, or would the team have to start from scratch \citep{avelino2016}.} Recorded (written or digitally recorded) knowledge can be misplaced, or employees may fail to record this knowledge in the first place. Technology can become irrelevant over time, and with it the knowledge thereof. Finally, social networks decay \citep{Argote1990,Argote2013_3,Thompson2007}. 

\subsection{Disruption and Knowledge Depreciation}

Organizational forgetting works in the opposite direction of the well-established learning curve. \citet{Benkard2000} showcases that the various production changes of the Lockheed L-1011 led to a (right-tilted) S-shape of the learning curve. Specifically, after production was cut in 1975, the unit cost began to increase, and they peaked as Lockheed hired additional staff to accelerate, motivated by the original learning curve that suggested that Lockheed could eventually sell the planes at cost \citep[pp. 60f.]{Argote2013_3}. The disruptions of regular production canceled out the positive effect from learning. Lockheed's production of the L-1011 describes a literal case of production disruption, but the disruption can take other forms, too, that lead to a retrogression of the learning curve. Economic cycles of up and downturns can disrupt organizational learning and lead to knowledge loss \citep{Rockart2019}. New technology, besides requiring new learning as laid out above, can  act as a disruptor of tried-and-true routines. After this technology is introduced organizations can struggle to adjust, and some organization develop entirely new routines and knowledge in response to new technology \citep{Amburgey1993,Edmondson2001}. Disruption can stem from inside the organization (e.g., restructuring/reorganization and layoffs), or from outside the organization (e.g., natural disasters) \citep{Anderson2014}.