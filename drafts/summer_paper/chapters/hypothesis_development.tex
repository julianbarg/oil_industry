\section{Pipelines Operators: Hierarchy, Disruption, and Learning}

The pipeline industry offers a context where the external environment is relatively constant. Extraction companies make long-term investments in different oil fields, and use mature instruments to estimate the available oil reserves. Based on these exploration activities, pipeline operators (or the extraction companies themselves) construct their infrastructure; the assets in their inventory hence change little over time. Compared e.g., consumer businesses that try to fit to a developing urban environment, the immediate physical and social environment of the physical assets is very stable (safe for natural disasters), and unlike the manufacturing sector, the product, and in many cases the customers (e.g., when a pipeline is the only one to supply a population center with oil or gas) are also constants. With the advent of renewable energy, prices has come under pressure, but the sector has secured political support from the US government and prices have, as of yet, always rebound.

Because of the stability of the pipeline industry's environment, we can isolate disruption exceptionally well; specifically, disruption is handed down from strategic initiatives at the top, and change the scale of the organization. As mentioned in the introduction, the overarching organization (be it an extraction company with pipeline assets, or a specialized pipeline operator) has an aspiration level for performance, and may adjust its strategy when the organization underperforms. These strategical adjustments disrupt the learning process of the business unit in charge of pipeline safety. Construction of new pipelines alter the scale and geographic scope of the organization, and entail the hiring and training of new staff. M\&As bring with them an overhaul of the hierarchy, and disrupt existing networks (e.g., when the resources of key personnel is stretched thin). 

\textbf{Hypothesis 1}: \textit{Strategic adjustments from the parent organization disrupt the learning of suborganizations.}

When the parent organization acts as a source of disruption for the business unit in charge of pipeline safety, we expect learning to be interrupted. On the other hand, when expansion (or standstill/contraction) comes to a more even and predictable pace, suborganizations will have the chance to learn and reduce unit prices or improve performance; in this case pipeline safety. This process can be likened to disrupted learning, as introduced by \citet{Rockart2019}, or oscillation between forgetting and learning \citep{Haunschild2015}: from the perspective of the suborganization, learning is interrupted by an internal source of disruption; from a more encompassing, organization-wide perspective (if a company has phases of strategic change and more constant phases) the organization oscillates between phases that allow for learning, and phases of pursuit of the overall aspiration level, which entails organizational forgetting on other levels of the hierarchy. 

The more "quiet" phases allow for departments to "digest" preceding changes e.g., through vicarious learning. A new pipeline segment, or an acquired asset offers many opportunities for vicarious learning, even if the the new asset differs in age, construction method, diameter etc. Initially, the new scope of the organization means that existing routines need to be adjusted. But once the department is familiar with the new assets, it may be able to adjust e.g., existing inspection schedules and procedures, and change schedules and assignments to optimize use of resources \citep{Huber1991}. When in the new setup, some mistakes, or new-mistakes occur, these function as signals for the organization as to how resources should now be allocated \citep{Baum2007, Desai2015}.

\textbf{Hypthesis 2}: \textit{Phases of stability, with less strategic adjustments being initiated by the parent organization, allow for the suborganizations to learn.}

Rather than disrupting learning, strategic adjustments from the parent organizations could also act as external shocks, that can spur additional learning. To illustrate: if this hypothesis, and hypothesis 2 both hold true, an organization might want to oscillate between strategic adjustments, and phases of stability, in order to maximize learning.

\textbf{Competing hypothesis 1c}: \textit{Strategic adjustments might act as external shocks to suborganization, and induce additional learning.}