\section{Introduction}
We observe organizational learning in the context of pipeline operators. The task and large parts of the environment (technology and immediate surrounding) of pipeline operators compared to other industries are very stable. The industry is yet to be disrupted by renewable energy in a major way. The only major source of disruptions are internal adjustments to strategy, such as the construction of a new pipeline, or the acquisition of a competitor. In the 2010s constructed almost a hundred thousand miles of new pipelines (as of 2018), in response to the fracking boom of the US oil and gas sector. In 2018, 406 pipeline incidents occurred, 151 of which were classified by the Federal Energy Regulatory Commission (FERC) as "significant". We track how companies handle their pipeline system extensions, and how the mode of development (i.e, stable or with many strategic adjustments) affects their incident rate.

The organizational goal of a pipeline operator (or extraction company with pipeline assets) is not identical with that of its suborganizations: the parent organization may respond to the business case for new pipelines, or pipeline acquisitions, for instance when new oil fields are discovered. The right portfolio of pipelines at the right time allows the organization to maximize profits. The department that is in charge of pipeline safety does not contribute to revenue generation - it is "sunk cost" in the best case, and a drain on profits when incidents occur. In the worst case, a streak of bad enough oil spills can threaten the existence of a smaller operator, such as HVI Cat Canyon: this small pipeline operator went bankrupt in 2019 after a number of pipeline leaks (in California, where laws are relatively stringent) led to a series of lawsuits; the organization could not afford to pay the fines \citep{WSJ2019}. Learning in this context occurs on separate organizational levels. \citet{Cyert1992} explains that an organization's \textit{aspiration level} drives activity at the organizational level; for instance, a company in the oil industry perceiving a shortfall in performance relative to its reference group might acquire new oil fields or construct a new pipeline to reach a new market. Pipeline safety at this level is a \textit{hygiene factor}: a certain level of performance is required for survival, but beyond that level, an improved pipeline safety does not make a large difference for competing in the market. Most importantly, pipeline safety is left censored: once you have achieved perfect safety, additional effort is wasted; further, the cost of making additional progress beyond a certain point by controlling for all eventualities might increase exponentially; hence, the organization may have an economic interest of not over-optimizing performance in this are (since once perfect safety is achieved, the additional effort does not yield any return).

This focus (on overall revenue or profit growth) constitutes a significant difference from the goals of the business unit that operates the pipelines. An oil spill can threaten careers in and the success of this department, while the department and its members cannot distinguish themselves through outstanding performance and achievements. The best case scenario for the pipeline safety is not to stand out excellent performance in an initiative, but to stand out through from the lack of anything happening. Hence, this department's interest is to prevent incidents. The safety track record is related to the overall organizational goal (of survival, or attaining the aspiration level), but it is not identical with the overarching organizational goal - pipeline safety is a subordinate goal. When given the opportunity, a department can learn and make strides toward its subordinate goal: the learning curve applies for a goal when an organization can gain experience by carrying out the same task repeatedly \citep{Argote2013_1}. When the organization is disrupted however (for instance because the parent organization implements changes), the learning process may be interrupted, and existing knowledge can become irrelevant or depreciate \citep[pp. 62 ff.]{Argote2013_3}. The learning curve is "reset".

In the context of pipeline operations, we test the relationship between parent organization activities and the suborganization's learning toward their subordinate organizational goal. The resetting of the learning process suggests that in time periods of stability, when the parent organization's activities are relatively constant, the department in charge of pipeline safety can engage in learning toward its subordinate organizational goal. When a big change occurs as the result of the parent organization's activities (e.g., the commissioning of a new pipeline), the department learning curve is disrupted and the department experiences a set back on the learning curve (now learning toward the operating safety of the overall, new system), thus, activities are less optimized and errors are more likely.

To measure disturbances from the overall organization, we turn to expansions of the pipeline network. The context of pipeline operators provides a unique research opportunity: US regulations mandate that organizations provide data on their pipeline network annually, and for FERC to provide this data to the public. Thus, we know the exact scope of pipeline operators across time. For our purpose this dataset is more suitable than accounting data, which can be less comparable across years or organizations due to different ways of depreciating assets, boosting revenue, etc. We measure how even an operator's rate of change is across years - we expect time periods of relatively steady rate of change to allow for suborganizations to learn and drive down incident rates. Incident counts are also provided by FERC, to which companies are obliged to report oil spills that either (1) led to a loss of 19l of liquid or more, (2) resulted in a personal injury, or (3) cause a property damage of \$50,000 or more.